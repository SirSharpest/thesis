% Created 2019-04-19 Fri 18:56
% Intended LaTeX compiler: pdflatex
\documentclass[11pt]{report}
\usepackage[utf8]{inputenc}
\usepackage[T1]{fontenc}
\usepackage{graphicx}
\usepackage{grffile}
\usepackage{longtable}
\usepackage{wrapfig}
\usepackage{rotating}
\usepackage[normalem]{ulem}
\usepackage{amsmath}
\usepackage{textcomp}
\usepackage{amssymb}
\usepackage{capt-of}
\usepackage{minted}
\renewcommand\maketitle{}
\usepackage[margin=0.8in]{geometry}
\usepackage{amssymb,amsmath}
\usepackage{fancyhdr} %For headers and footers
\pagestyle{fancy} %For headers and footers
\fancyhead{}
\fancyhead[L, RO]{\rightmark}
\fancyhead[R, LO]{\thepage}
\fancyfoot{}
\usepackage{lastpage} %For getting page x of y
\usepackage{float} %Allows the figures to be positioned and formatted nicely
\restylefloat{figure} %and this command
\usepackage{titlesec}
\setcounter{secnumdepth}{4}
\usepackage{minted}
\setminted{frame=single,framesep=10pt}
\usepackage[parfill]{parskip}
\usepackage{titlesec}
\usepackage{tabularx}
\usepackage{multicol}
\usepackage{array}
\usepackage[page,toc,titletoc,title]{appendix}\usepackage{subfig}
\usepackage[subfigure]{tocloft}
\newcolumntype{C}{ >{\centering\arraybackslash} m{10.4cm} }
\usepackage[usenames, dvipsnames]{color}
\usepackage{pdfpages}
\usepackage[nocompress]{cite}
\usepackage{titling}
\usepackage{lipsum}
\usepackage[super]{nth}
\usepackage[numbers, sort&compress]{natbib}
\usepackage[font={small}, labelfont=bf]{caption}
\renewcommand{\bibname}{References}
\usepackage{framed}
\usepackage{aliascnt}
\usepackage{etoolbox}
\usepackage{aliascnt}
\usepackage{notoccite}
\usepackage{mathtools, cases}
\usepackage{subfiles}
\usepackage{hyperref}
\hypersetup{colorlinks=true,linkcolor=black, citecolor=black}
\date{}
\title{\textbf{Computational modeling of information transfer between cells}}
\hypersetup{
 pdfauthor={Nathan Hughes},
 pdftitle={\textbf{Computational modeling of information transfer between cells}},
 pdfkeywords={},
 pdfsubject={},
 pdfcreator={Emacs 26.1 (Org mode 9.1.9)},
 pdflang={English}}



\begin{document}

\newaliascnt{eqfloat}{equation}
\newfloat{eqfloat}{h}{eqflts}
\floatname{eqfloat}{Equation}

\newcommand*{\ORGeqfloat}{}
\let\ORGeqfloat\eqfloat
\def\eqfloat{%
  \let\ORIGINALcaption\caption
  \def\caption{%
    \addtocounter{equation}{-1}%
    \ORIGINALcaption
  }%
  \ORGeqfloat
}

\newcommand{\zebra}[3]{%
    {\realnumberstyle{#3}}%
    \begingroup
    \lst@basicstyle
    \ifodd\value{lstnumber}%
        \color{#1}%
    \else
        \color{#2}%
    \fi
        \rlap{\hspace*{\lst@numbersep}%
        \color@block{\linewidth}{\ht\strutbox}{\dp\strutbox}%
        }%
    \endgroup
}

\newcommand{\listequationsname}{List of Equations}
\newlistof{myequations}{equ}{\listequationsname}
\newcommand{\myequations}[1]{%
\addcontentsline{equ}{myequations}{\protect\numberline{\theequation}#1}\par}
\setlength{\cftmyequationsnumwidth}{2.5em}% Width of equation number in List of Equations

\hyphenpenalty=10000



\titleformat{\chapter}[display]
   {\normalfont\huge\bfseries}{\chaptertitlename\ \thechapter}{20pt}{\Huge}
\titlespacing*{\chapter}{10pt}{10pt}{10pt}


\thispagestyle{empty}
\renewcommand{\headrulewidth}{0pt}

\begin{titlepage}
    \begin{center}
        \vspace*{1cm}

        \Huge
        \textbf{Computational modelling of information transfer between cells}

        \vspace{0.5cm}
        \Large

        A report presented as part of the\\PhD year 2 review, May 2020

        \vspace{1.5cm}

        \textbf{Nathan Hughes}

        \vfill

        \vspace{0.8cm}

        \Large
        Computational Systems Biology\\
        John Innes Centre, Norwich

    \end{center}

\end{titlepage}

\thispagestyle{empty}
\begin{abstract}

  For plants, efficient cell to cell communication is vital as unlike animals
  they lack a centralised information processing system. Additionally, typical
  plant cells are non-mobile and are encased in a cell wall. Therefore cells
  must have mechanisms for distributing information to relevant destinations
  that can move through the cell wall barrier. Symplastic channels known as
  plasmodesmata facilitate this intercellular pathway. Plasmodesmata, more than
  being simple passive channels have the ability to alter their aperture and so
  influence the degree of molecular traffic between cells. These alterations in
  intercellular flux are known to be used by plants to coordinate development,
  as well as to react to both biotic and abiotic stresses. Thus, it is essential
  that we enhance our mechanistic understanding of signalling between cells if
  we want to better understand how plants adapt to change. This project focuses
  on the specific mechanisms plants employ in response to pathogen attack.
  Through mathematical and computational simulations we seek to unravel the role
  of plasmodesmata in altering defence signals and how encoded signals can
  subsequently be altered to have different effects throughout defence.


\end{abstract}


\clearpage
\renewcommand{\headrulewidth}{1pt}

\clearpage

\clearpage
\tableofcontents
\thispagestyle{plain}          
\clearpage
\listoffigures
\clearpage
\listofmyequations
\clearpage
\listoftables
\clearpage


\subfile{./chapters/introduction.tex}
\subfile{./chapters/diffusion.tex}
\subfile{./chapters/narrowescape.tex}
\subfile{./chapters/networking.tex}
\subfile{./chapters/rnaseq.tex}
\subfile{./chapters/gendisscusion.tex}
\subfile{./chapters/appendix.tex}

% \subfile{./chapters/projectaims.tex} % Removing aims, as each chapter should
% address this 

\clearpage
\renewcommand{\bibname}{Bibliography}
\addcontentsline{toc}{chapter}{Bibliography}
\bibliography{library}
\bibliographystyle{unsrtnat}
\end{document}


\end{document}


%%% Local Variables:
%%% TeX-command-extra-options: "-shell-escape"
%%% mode: latex
%%% TeX-master: t
%%% End:
